%% Exemple de source LaTeX pour un article soumis à TALN
\documentclass[10pt,a4paper,twoside]{article}

\usepackage{times}
\usepackage[utf8]{inputenc}
\usepackage[T1]{fontenc}
\usepackage{graphicx}

% faire les \usepackage dont vous avez besoin avant le \usepackage{taln2010} 

\usepackage{taln2011}
\usepackage[frenchb]{babel}

\title{Modèle de document pour TALN 2011}

\author{Untel Trucmuche\up{1, 2}\quad Unetelle Machinchose\up{1, 3}\\
  (1) LIRMM, 161, rue ADA  34392 Montpellier Cedex 5\\ 
  (2) INSERM, U729, 75006 Paris \\ 
  (3) INALCO, CRIM, 75343 Paris Cedex 07 \\ 
  utrucmuche@lirmm.fr, umachinchose@inserm.fr \\ 
}

\fancyhead[CO]{Modèle de document pour TALN 2011} 
\fancyhead[CE]{Untel Trucmuche, Unetelle Machinchose} 

\begin{document}

\maketitle

\resume{
Ici, un résumé en français (max. 150 mots).
}

\abstract{
Here an abstract in English (max. 150 words).
}

\motsClefs{Ici une liste de mots-clés en français}
{Here a list of keywords in English}

%% Aller à la page suivante si nécessaire
\newpage
%%================================================================
\section{Titre de la première partie}

Les articles soumis ne devront pas dépasser 12 pages en Times 10, espacement simple, soit environ 4500 mots, figures, exemples et références compris. Les propositions de démonstrations ne devront pas dépasser 3 pages. Les articles seront rédigés en français pour les francophones, en anglais pour ceux qui ne maîtrisent pas le français. Les versions devront être en format A4. Prévoir des marges de 1,27 cm à gauche, à droite, en haut et en bas (marges étroites).\\
Une feuille de style LaTeX et un modèle Word sont disponibles sur le site web de la conférence. Le site web de la conférence prévoit un formulaire interactif pour la soumission des articles, à télécharger au format PDF. \\
Les versions finales devront être envoyées en format pdf. Le message d’acceptation de la proposition indiquera l’adresse courriel pour l’envoi de la version définitive.\\
La première page ne contient que le titre, le nom de l’auteur, son affiliation, les résumés et les mots-clés, alors que le corps de l’article commence à la deuxième page. Cependant, au besoin, on fera commencer le corps de l’article directement après les mots-clés, en veillant à ce qu’un titre ne soit pas séparé du paragraphe suivant. 

\subsection{Titre de la première sous-partie}
\begin{itemize}
\item Une liste à puce 
\end{itemize}
\begin{enumerate}
\item Une liste numérotée
\end{enumerate}

\begin{table}[!h]
\centering
	\begin{tabular}{|c|p{4cm}|}
	\hline
	Un tableau&\\
	\hline
	&Les cellules ainsi que le tableau sont centrés\\
	\hline
	\end{tabular}
\caption{Un tableau}
\end{table}

\begin{figure}[htbp] 
\begin{center} 
\includegraphics{atala.png}
\end{center} 
\caption{Une image comme figure} \label{image} \
\end{figure}


Une note de bas de page\footnote{Que voici !}.\\
Le renvoi à la bibliographie : \cite{Bernhard07}

\begin{figure}[htbp] 
\begin{center} 
~\\
~\\
\framebox[5cm]{étape 1}\\
 ~~~~~~~~ | \\
 ~~~~~~~~ | \\
\framebox[5cm]{étape 2}\\
~~~~~~~~ | \\
~~~~~~~~ | \\
\framebox[5cm]{étape 3}\\
~~~~~~~~ | \\
~~~~~~~~ | \\
\framebox[5cm]{étape 4}\\

\end{center} 
\caption{Un schéma comme figure} \label{schema} \
\end{figure}

%%================================================================
\subsection{Titre de la deuxième sous-partie}

etc.

\section{Titre de la deuxième partie}

etc.

%%================================================================
\section*{Remerciements (pas de numéro)} 

Paragraphe facultatif

%%================================================================
%% Note : si l'on préfère éviter de factoriser les crossrefs :
%% bibtex -min-crossrefs 99 taln-exemple
%%================================================================
\bibliographystyle{taln2002}
\bibliography{biblio}
\nocite{TALN2007,LaigneletRioult09,LanglaisPatry07,SeretanWehrli07
}

%%================================================================
\end{document}
